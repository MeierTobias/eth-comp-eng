\section{User Mode}

\subsection{Process}

A process is an abstraction for virtualizing resources in a computing system and it provides an application with the illusion that
\begin{itemize}
    \item the entire memory is available.
    \item the CPU is purely dedicated.
    \item I/O resources are purely dedicated.
\end{itemize}
This is achieved with the so-called sandbox execution.

\subsection{Sandbox Execution}
The basic idea is that processes run without any interference from the kernel unless needed like in the case of:
\begin{itemize}
    \item explicit service needed
          \begin{itemize}
              \item Allocate pages for a new heap
              \item Print a character on the console
          \end{itemize}
    \item implicit service needed
          \begin{itemize}
              \item Soft page fault (e.g., demand paging)
          \end{itemize}
    \item force stop/continue
          \begin{itemize}
              \item Run a different process
              \item Hard page fault (i.e., application bug)
          \end{itemize}
\end{itemize}

To achieve this there are at least two modes:
\begin{itemize}
    \item Processes running in the sandbox (user mode)
    \item Kernel running in outside the sandbox where all instructions are allowed (kernel mode)
\end{itemize}

\subsubsection{Address Spaces}
Each process (user-mode) runs in its own virtual address space. The kernel can either run in a separate (isolated kernel) virtual address space or in the same virtual address space as the process (joint kernel).

\paragraph{Isolated Kernel}

\includegraphics[width=\linewidth]{user_mode_isolated_kernel.png}

An isolated kernel can provide certain security benefits but the system has to switch the address space (and privilege) every time a process needs a service from the kernel which is expensive.

\paragraph{Joint Kernel}

\includegraphics[width=\linewidth]{user_mode_joint_kernel.png}

A kernel service request only requires a switching of the privilege and not the VMA but now the privilege have to be associated to different areas within the address space. This is done by setting the \textbf{U-bit} in the page table entry.

Data pages with U-bit unset (kernel data pages) can only be accessed by code run in pages with U-bit unset (kernel code pages). This means only kernel code can access kernel data.

\includegraphics[width=\linewidth]{user_mode_joint_kernel_privileges.png}

While the kernel boots (before enabling the MMU) the U-bit in the page table entries that map the kernel should be set to 0. When starting a new process, the U-bit is set to 1 for the page table entries that map its user-mode VMAs.

\subsection{Privilege Levels}

RISC-V has three different privilege levels:

\newpar{}
\ptitle{Machine Mode}

Machine mode is the only mandatory mode in RISC-V. All RISC-V CPUs boot in machine mode. Machine mode has access to all hardware features but does not have an MMU.

\newpar{}
\ptitle{Supervisor Mode}

The kernel runs in the supervisor mode. Once in supervisor mode, the kernel can enable the MMU. A kernel (e.g., Jake) starting in machine mode can
empower the supervisor mode by \textbf{delegating} (most) exceptions (e.g., page fault) before switching to it.

\newpar{}
\ptitle{User Mode}

The RISC-V mode with sandboxed execution. The kernel switches to this mode every time it wants to run an application. Certain operations are not allowed (e.g., writing to \texttt{satp}).

\subsubsection{Switching to Lower Privileges in RISC-V}

\includegraphics[width=\linewidth]{user_mode_switch_privileges.png}

\newpar{}
\ptitle{Terminology Example: Returning From S to U}
\begin{itemize}
    \item SPP: Supervisor Previous Privilege = 0. This means that, returning (via \texttt{sret}) from S, the previous privilege was the user mode. When ``returning'' from S to S this would be 1.
    \item PC = SEPC: Supervisor Exception PC. This means that, returning (via \texttt{sret}) from S, the usermode app will continue at SP=SEPC.
\end{itemize}


\subsubsection{Switching to higher privilege IMPLICITLY}

This happens when an exception occurs (see Section~\ref{exceptions} for exception handling). Exceptions are by default delivered to machine mode. Jake (and other kernels) delegates user-mode exceptions from machine mode to supervisor mode. 

\subsubsection{Switching to higher privilege EXPLICITLY}

If the user-mode process requests a service in a controlled manner it is called a \textbf{system call} for example
\begin{itemize}
    \item Printing a character to the screen.
    \item Allocating a new heap (VMA).
    \item Creating a new process.
\end{itemize}

RISC-V features an instruction named \texttt{ecall} for this purpose.\ \texttt{ecall} behaves like a user-initiated exception. There is a contract between user-mode applications and the supervisor-mode kernel for requesting service.

\newpar{}
% TODO: Supervisor mode to machine mode (next lecture)

% TODO: Interrupts (next lecture more)