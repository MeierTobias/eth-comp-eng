\section{CPU Design}\label{cpu design}

\begin{center}
    \includegraphics[width=\linewidth]{cpu_flowchart.png}
\end{center}

\newpar{}
\ptitle{Remarks}
\begin{itemize}
    \item 12 bits $\to$ \texttt{Imm Gen} $\to$ 32 bits (extends sign)
    \item \texttt{MemRead}/\texttt{MemWrite} are \textbf{both} needed as some instructions don't use data memory at all.
          \begin{itemize}
              \item Don't just use one wire to switch between 0/1
              \item Use 2 wires. If both =0: Turn power off when it is not required.
          \end{itemize}
    \item \texttt{RegWrite} blocks writing if needed (e.g.\ for branches no register write is allowed)
\end{itemize}

\subsection{Opcode}

\begin{footnotesize}
    \renewcommand{\arraystretch}{1.2}
    \setlength{\oldtabcolsep}{\tabcolsep}\setlength\tabcolsep{6pt}

    \begin{tabularx}{\linewidth}{@{}T T Z Z Z Z Z Z T @{}}\label{tab:opcode}
                                                               & \multicolumn{5}{c}{\fncode{Opcode}} &          & \multicolumn{2}{c}{\color{teal}\fncode{Pipelining}}                                                                                       \\
        \cmidrule{2-6}\cmidrule{8-9}
                                                               & Signal name                         & R-format & lw                                                  & sw & beq &   & \color{teal}Stage                                              & NOP \\
        \cmidrule{2-6}
        \multirow{7}{*}{\begin{sideways}Input\end{sideways}}   & I[6]                                & 0        & 0                                                   & 0  & 1   &                                                                          \\
                                                               & I[5]                                & 1        & 0                                                   & 1  & 1   &                                                                          \\
                                                               & I[4]                                & 1        & 0                                                   & 0  & 0   &                                                                          \\
                                                               & I[3]                                & 0        & 0                                                   & 0  & 0   &                                                                          \\
                                                               & I[2]                                & 0        & 0                                                   & 0  & 0   &                                                                          \\
                                                               & I[1]                                & 1        & 1                                                   & 1  & 1   &                                                                          \\
                                                               & I[0]                                & 1        & 1                                                   & 1  & 1   &                                                                          \\
        \cmidrule{2-6}
        \morecmidrules\cmidrule{2-6}
        \multirow{10}{*}{\begin{sideways}Output\end{sideways}} & ALUSrc                              & 0        & 1                                                   & 1  & 0   &   & \multirow{3}{*}{\begin{sideways}\color{teal}EX\end{sideways}}        \\
                                                               & ALUOp1                              & 1        & 0                                                   & 0  & 0   &   &                                                                      \\
                                                               & ALUOp0                              & 0        & 0                                                   & 0  & 1   &   &                                                                      \\
        \cmidrule{2-6}
                                                               & Branch                              & 0        & 0                                                   & 0  & 1   &   & \multirow{3}{*}{\begin{sideways}\color{teal}MEM\end{sideways}} & 0   \\
                                                               & MemWrite                            & 0        & 0                                                   & 1  & 0   &   &                                                                & 0   \\
                                                               & MemRead                             & 0        & 1                                                   & 0  & 0   &   &                                                                & 0   \\
        \cmidrule{2-6}
                                                               & MemtoReg                            & 0        & 1                                                   & X  & X   &   & \multirow{3}{*}{\begin{sideways}\color{teal}WB\end{sideways}}  &     \\
                                                               & RegWrite                            & 1        & 1                                                   & 0  & 0   &   &                                                                & 0   \\
    \end{tabularx}

    \renewcommand{\arraystretch}{1}
    \setlength{\tabcolsep}{\oldtabcolsep}
\end{footnotesize}
\newpar{}
\code{X}: Don't care

\newpar{}
\ptitle{Remarks}
\begin{itemize}
    % \item These results can all be achieved using the \texttt{opcode} bits only. % 
    \item Using one multiplexer per control signal: $2^n$ input bits.
          \begin{itemize}
              \item Use Karnaugh or Quine-McCluskey for optimization.
          \end{itemize}
\end{itemize}

\subsubsection{ALUOp}
\begin{footnotesize}
    \renewcommand{\arraystretch}{1.2}
    \setlength{\oldtabcolsep}{\tabcolsep}\setlength\tabcolsep{6pt}

    \begin{tabularx}{\linewidth}{@{}T Z Z Z T Z@{}}
        Opcode                  & ALUOp               & Funct7        & Funct3 & ALU action                                              & ALU contr         \\
        \cmidrule{2-6}
        lw                      & 00                  & not available & XXX    & add                                                     & 0010              \\
        sw                      & 00                  & not available & XXX    & add                                                     & 0010              \\
        beq                     & 01                  & not available & XXX    & sub                                                     & 0110              \\
        \multirow{4}{*}{R-type} & \multirow{4}{*}{10} & 0000000       & 000    & \color{teal} add                                        & \color{teal} 0010 \\
                                &                     & 0100000       & 000    & \color{teal} sub                                        & \color{teal} 0110 \\
                                &                     & 0000000       & 111    & \color{teal} and                                        & \color{teal} 0000 \\
                                &                     & 0000000       & 110    & \color{teal} or                                         & \color{teal} 0001 \\
                                &                     &               &        & \multicolumn{2}{c}{\color{teal} \tt{ALU control lines}}                     \\
    \end{tabularx}

    \renewcommand{\arraystretch}{1}
    \setlength{\tabcolsep}{\oldtabcolsep}
\end{footnotesize}
\newpar{}
\ptitle{Remarks}
\begin{itemize}
    \item The \code{ALUOp} bits are part of the Opcode
          % \item Note however, that these bits can be identical for multiple instructions.
    \item Furthermore, the \texttt{funct} bits define the \texttt{ALU contr} bits which steer the exact operation performed by the ALU.
    \item Also note that the \texttt{Funct3} bits are don't care e.g.\ for \texttt{lw} but only w.r.t.\ the ALU.
          \begin{itemize}
              \item These bits define e.g.\ the exact load size such as byte, half word.
          \end{itemize}
          % \item \texttt{Funct7} are only needed for R-type.
\end{itemize}

\subsection{Parallel Instruction Execution}

To improve the performance of the single-cycle CPU the idle time of each component is reduced  $\to$ increase \textit{throughput}. This can be achieved by:

\begin{itemize}
    \item \textbf{Reuse} of logic units for different tasks (instructions), e.g.\ use ALU for address/branch calculations.
    \item \textbf{Pipelining:} Executing multiple instructions at the same time (see~\ref{pipelining}).
    \item \textbf{Multi-issue processors} that fetch multiple instructions in each cycle (see~\ref{multi-issue processors}).
\end{itemize}

\textbf{Remark}

Alternatively, the \textit{latency} i.e.\ the instruction length could be reduced too.

\subsection{Multi-Issue Processors}\label{multi-issue processors}
Multi-issue processors fetch multiple instructions in each cycle and try to execute them in parallel.
\begin{itemize}
    \item This can be achieved by adding more (independent) pipelines, ALUs etc.
    \item As a result, multiple instructions per cycle (IPC) can be executed (IPC > 1)
\end{itemize}
\subsubsection{Static Multi-Issue Processors}
The CPU fetches a \textbf{static number} of instructions at each cycle. AKA VLIW (Very Long Instruction Word)

\includegraphics[width=\linewidth]{cpu_static_multi-issue_processors.png}

The compiler ensures that all instructions fetched in the same cycle (packet) do not have any hazards with each other (no int\textcolor{red}{ra}-packet hazards). The CPU detects int\textcolor{red}{er}-packet hazards and handles them as needed.

\subsubsection{Dynamic Multi-Issue Processors (Superscalar)}
The processor decides how many instructions to fetch in each cycle. This depends on the available resources and hazards.
\newpar{}
\ptitle{Out-of-Order (OoO) Instruction Execution}

OoO instruction execution allows to avoid certain hazards but requires a tracking of the dependencies between the instructions.
\begin{center}
    \includegraphics[width=.8\linewidth]{cpu_multi-issue_processors_OoO.png}
\end{center}