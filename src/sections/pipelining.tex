\section{Pipelining}\label{pipelining}

\includegraphics[width=\linewidth]{cpu_pipelining.png}

The CPU design is segmented into five sub-steps. Each segment performs an action of a different instruction.

\subsection{Pipeline Registers}
Pipeline registers allow holding temporary information about different stages and forwarding necessary information in the pipeline. They are internal to the CPU and cannot be accessed from software. Each instruction propagates through the pipeline as data and control information.
\subsubsection{Data Registers}
\begin{center}
    \includegraphics[width=\linewidth]{cpu_pipeline_datareg.png}
\end{center}
Examples of data that are commonly preserved using pipeline registers are
\begin{enumerate}
    \item PC for later PC-relative branch. Branch outcome known first after register stage (assuming shortcut:~\ref{shortcut}).\\
          Instruction word. Also contains indices of registers used per instruction (needed e.g.\ for hazard detection or final WB).
          % TODO: Razavi told in lecture that the entire instruction word is pushed. Can we assume that the opcode and funct bits are thrown away after the IF/ID stage? At this point, we calculated the control bits and only need the register indices and immediates for later.
    \item Data read from register file (for EX \textit{and} MEM stage).
    \item ALU results (Used as address for MEM or as data for WB).
    \item Buffer for MEM/WB (yields 5 short stages instead of 4 long ones).
\end{enumerate}

\subsubsection{Control Registers}
Instead of calculating control bits on the fly, at the ID stage the \textbf{per-stage control} for \textit{all} stages of the decoded instruction is calculated.
The control bits of the individual stages are then stored in an extension of the pipeline registers according to opcode in Table~\ref{tab:opcode}.
\begin{itemize}
    \item[+] shorter clock cycle
    \item[+] fewer bits in the pipeline registers (instruction not passed along)
    \item[-] redundant control information
          % TODO: Which bits are redundant? Are the MEM&WB bits meant (as they are also available in the ID/EX stage. If we don't push them we loose them so they are not really redundant in my opinion.)? 
\end{itemize}

\begin{center}
    \includegraphics[width=\linewidth]{cpu_pipelining_controlreg.png}
\end{center}

Three stages need control bits
\begin{enumerate}
    \item EX: \code{ALUSrc, ALUOp}
    \item MEM: \code{Branch, MemWrite, MemRead}
    \item WB: \code{MemtoReg, RegWrite}
\end{enumerate}

\subsection{Hazards}
Dependencies between instructions produce three types of \textbf{Hazards}:
\begin{itemize}
    \item~\ref{structural hazards} Structural Hazards
    \item~\ref{data hazards} Data Hazards
    \item~\ref{control hazards} Control Hazards
\end{itemize}

\subsubsection{Structural Hazards}\label{structural hazards}

\includegraphics[width=\linewidth]{cpu_pipelining_structural_hazard.png}

Both steps need to access the memory. To fix this, individual resources can be added for the instruction memory and the data memory tasks.

\begin{center}
    \includegraphics[width=0.3\linewidth]{cpu_pipelining_structural_hazard_fix.png}
\end{center}

\subsubsection{Data Hazards}\label{data hazards}

A subsequent instruction needs data or a register that is not yet ready.

\begin{itemize}
    \item \textbf{RAW:\ read after write} (true dependency)
          \begin{lstlisting}[escapechar=@]
add @\aftergroup\magentacodecolor@t1@\aftergroup\blackcodecolor@, t2, t3
sub t4, @\aftergroup\magentacodecolor@t1@\aftergroup\blackcodecolor@, t5
\end{lstlisting}
    \item \textbf{WAR:\ write after read} (anti dependency)
          \begin{lstlisting}[escapechar=@]
add t1, @\aftergroup\magentacodecolor@t2@\aftergroup\blackcodecolor@, t3
sub @\aftergroup\magentacodecolor@t2@\aftergroup\blackcodecolor@, t4, t5
\end{lstlisting}
    \item \textbf{WAW:\ write after write} (output dependency)
          \begin{lstlisting}[escapechar=@]
lw @\aftergroup\magentacodecolor@t1@\aftergroup\blackcodecolor@, 100(t2)
lw @\aftergroup\magentacodecolor@t1@\aftergroup\blackcodecolor@, 200(t4)
\end{lstlisting}
\end{itemize}

The \textbf{WAR} and \textbf{WAW} hazard does not apply to the design showed at the beginning of this section. This is only an issue for out-of-order CPU's.

\paragraph{Detecting RAW hazards in the Pipeline}

The pipeline registers are used to check if a current \code{src} register (\code{rs1/rs2}) is found in any earlier instruction/later pipeline register.
\begin{lstlisting}
PipelineID/EX.rs1/2 == PipelineEX/MEM.rd
PipelineID/EX.rs1/2 == PipelineMEM/WB.rd
\end{lstlisting}

\newpar{}
Depending on the relative order, either the \code{EX/MEM} or the \code{MEM/WB} pipeline register has to be checked.

\begin{center}
    \includegraphics[width=0.8\linewidth]{example_hazard_detection.png}
\end{center}

\newpar{}
\ptitle{Remarks}
\begin{itemize}
    \item This approach can lead to unnecessary forwarding when using store operations (already stored register is still detected in \code{PipelineMEM/WB.rd}) % TODO: store has no rd
    \item 3 components of hardware logic are required to handle data hazards:
          \begin{enumerate}
              \item Hazard detection logic (detect that there \textit{actually is} a hazard)
              \item Forwarding logic
              \item Stalling logic
          \end{enumerate}
\end{itemize}

\paragraph{RAW Hazard Types}
\renewcommand{\arraystretch}{1.3}
\setlength{\oldtabcolsep}{\tabcolsep}\setlength\tabcolsep{6pt}

\begin{tabularx}{\linewidth}{@{}lllX@{}}
    Forwarding                     & Stalls & Type & Comment                                      \\
    \midrule
    no                             & 2      & -    & assuming reg-write takes second half, else 3 \\
    \code{EX/MEM} $\to$ \code{EX}  & 0      & 1    & no memory load                               \\
    \code{MEM/WB} $\to$ \code{EX}  & 1      & 2    & memory load                                  \\
    \code{MEM/WB} $\to$ \code{MEM} & 0      & 2    & memory load                                  % TODO: not sure if this is right
\end{tabularx}

\renewcommand{\arraystretch}{1}
\setlength\tabcolsep{\oldtabcolsep}



\paragraph{Solution: Forwarding}
\textbf{Type 1} data hazards (not containing a memory load) can be resolved by forwarding data from an earlier instruction.

\newpar{}
For example, the data in the \code{EX/MEM} pipeline register can be forwarded to the \code{EX} stage.

\includegraphics[width=\linewidth]{cpu_pipelining_raw.png}

\ptitle{Forwarding Unit}
\begin{center}
    \includegraphics[width=.7\linewidth]{cpu_pipelining_forwarding_simple.png}
\end{center}
First the comparison
\begin{lstlisting}
    PipelineID/EX.rs1/2 == PipelineEX/MEM.rd
    PipelineID/EX.rs1/2 == PipelineMEM/WB.rd
\end{lstlisting}
is executed. Then, the ALU source operands are chosen based on the result:
\begin{footnotesize}
    \renewcommand{\arraystretch}{1.4}
    \setlength{\oldtabcolsep}{\tabcolsep}\setlength\tabcolsep{6pt}
    
    \begin{tabularx}{\linewidth}{@{}T Z Z X @{}}
                                                                                  & MUX Control   & Source & ALU operand from                            \\
        \cmidrule{2-4}
        \multirow{3}{*}{\begin{sideways}\begin{scriptsize}operand 1\end{scriptsize}\end{sideways}} & ForwardA = 00 & ID/EX  & From register file                     \\
                                                                                  & ForwardA = 10 & EX/MEM & From prior ALU result                  \\
                                                                                  & ForwardA = 01 & MEM/WB & From data memory or earlier ALU result \\[1.5em]
        \multirow{3}{*}{\begin{sideways}\begin{scriptsize}operand 2\end{scriptsize}\end{sideways}} & ForwardB = 00 & ID/EX  & From register file                     \\
                                                                                  & ForwardB = 10 & EX/MEM & From prior ALU result                  \\
                                                                                  & ForwardB = 01 & MEM/WB & From data memory or earlier ALU result \\
    \end{tabularx}
    
    \renewcommand{\arraystretch}{1}
    \setlength{\tabcolsep}{\oldtabcolsep}
\end{footnotesize}

\paragraph{Solution: Stalling}
For \textbf{Type 2} hazards (containing a memory load) which need forwarding to the \code{EX} stage, forwarding alone is not sufficient and one additional stall has to be added.

\includegraphics[width=\linewidth]{cpu_pipelining_raw_stall.png}

\newpar{}
As mentioned in Table~\ref{tab:opcode}, setting the four bits (\code{Branch, RegWrite, MemWrite, MemRead}) to zero creates and pushes a \code{NOP} instruction into the pipeline: \code{pc} stays, registers are not updated, memory is left untouched. To achieve this, the hazard detection unit selects the zero input of the multiplexer yielding \textbf{zero control registers}.\\
First the comparison
\begin{lstlisting}
(PipelineIF/ID.rs1/2 == PipelineID/EX.rd) && PipelineID/EX.MemRead
\end{lstlisting}
is executed. This is different to Type 1 RAW. Then, the hazard detection unit can push a \code{NOP} instruction if necessary.
\begin{center}
    \includegraphics[width=\linewidth]{cpu_pipelining_NOP.png}
\end{center}
Note that during the bubble
\begin{itemize}
    \item \code{pc} is preserved using \code{PCWrite}
    \item the pipeline registers are kept using \code{IF/IDWrite}
\end{itemize}

\subsubsection{Control Hazards}\label{control hazards}

A branch instruction needs to be fully evaluated in order to know which instruction we need to execute next.

\begin{lstlisting}[escapechar=@]
@\aftergroup\magentacodecolor@beq@\aftergroup\blackcodecolor@ t2, t5, 40
lw t6, 400(t0)
\end{lstlisting}

There are three ways to solve this problem
\begin{itemize}
    \item~\ref{contr_stalling} Stalling
    \item~\ref{shortcut} Shortcut
    \item~\ref{prediction} Prediction
\end{itemize}

\paragraph{Stalling}\label{contr_stalling}

Wait until the outcome is known.
\begin{itemize}
    \item[+] Simple design
    \item[$-$] Always two stalls
\end{itemize}

\includegraphics[width=\linewidth]{cpu_pipelining_control_hazard_bubble.png}

\paragraph{Shortcut}\label{shortcut}

Wait less by adding additional logic to calculate the branch outcome sooner.

\begin{itemize}
    \item[+] Simple design
    \item[$-$] Always one stall
    \item[$-$] Needs more logic
\end{itemize}

\includegraphics[width=\linewidth]{cpu_pipelining_control_hazard_shortcut_logic.png}

\includegraphics[width=\linewidth]{cpu_pipelining_control_hazard_shortcut.png}

\paragraph{Prediction}\label{prediction}
Predict the outcome of a branch instruction and roll back if the prediction was wrong.
\begin{itemize}
    \item The previous solutions waited for the branch outcome before the first action was taken. % TODO: This is exactly what was stated above and here. Could we remove these two bullet points?
          \begin{itemize}
              \item Now action is taken before the result is known.
          \end{itemize}
    \item In general:
          \begin{itemize}
              \item Simple prediction needs less HW but is less reliable.
              \item Advanced prediction needs complex HW but is more reliable.
          \end{itemize}
\end{itemize}

\newpar{}
\ptitle{Pipeline Flush}

Is a \textit{mechanism} to clear out the executed parts of the followup instruction if the prediction was wrong (or in case of an interrupt/ exception). This has to happen before the downstream instructions reach the MEM or WB stage (DRAM corruption). If the shortcut unit presented in~\ref{shortcut} detects a branch, the hazard detection unit flushes the wrongly fetched instruction and adds the immediate (given by the branch) to the PC yielding the right instruction in the next cycle (creates \textbf{1 bubble}).
\begin{center}
    \includegraphics[width=\linewidth]{cpu_pipeline_flush.png}
\end{center}
\newpar{}
\ptitle{Static Prediction}

Static prediction assumes that the branch is \textbf{not taken} and just continues with the next instruction. Due to the usage of loops this prediction is often \textbf{wrong}.
\begin{center}
    \includegraphics[width=\linewidth]{static_prediction.png}
\end{center}
The example assumes the shortcut logic~\ref{shortcut} to enable \code{IF/ID}$\to$\code{IF} forwarding (orange arrow).

\newpar{}
\ptitle{Dynamic Prediction}

The CPU learns which direction the branch takes. The prediction is based on the last (or multiple last) branch decision(s). %ChkTex 36
\begin{center}
    \includegraphics[width=.5\linewidth]{cpu_pipelining_control_hazard_dyn_pred.png}
\end{center}

\newpar{}
\ptitle{2-bit Dynamic Prediction}
\begin{center}
    \includegraphics[width=.5\linewidth]{cpu_pipelining_control_hazard_dyn_pred_2bit.png}
\end{center}

\begin{itemize}
    \item This method operates well on nested loops (repeated execution of the same loop).
    \item Correct prediction becomes more important the deeper the pipeline is.
\end{itemize}


