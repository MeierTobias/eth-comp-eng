\section{Assembler}
\subsection{Interaction with Memory}
The CPU uses 32-bit sized instructions to
\begin{enumerate}[leftmargin=20pt]
    \item read memory (DRAM) into registers if needed
    \item execute computation using registers
    \item write the changed registers back into memory (DRAM)
\end{enumerate}

\newpar{}
\textbf{Remark:} While the CPU always uses 32-bit sized instructions to access memory,
the address space can be larger:

\renewcommand{\arraystretch}{1.3}
\setlength\tabcolsep{6pt} % default value: 6pt
\begin{tabularx}{\linewidth}{@{}lll@{}}
    Name        & Register size & Address space          \\
    \cmidrule{1-3}
    \code{RV32} & 32 bit        & $2^{32} \simeq 4$ Gbit \\
    \code{RV64} & 64 bit        & $2^{64} \gg 1$ Pbit    \\
\end{tabularx}
\renewcommand{\arraystretch}{1}
\setlength\tabcolsep{6pt} % default value: 6pt

\subsection{Signedness}
$n$ bits can either represent [0,$2^{n}-1$] \textbf{unsigned} or [$-2^{n-1}, 2^{n-1}$] \textbf{signed} numbers.
For \textbf{signed} numbers the \textbf{two's complement}
\noindent\begin{equation*}
    -a_{n-1}\cdot2^{n-1}+\sum_{i=0}^{n-2}a_{i}\cdot 2^{i}
\end{equation*}
is used, which allows for addition and subtraction with only one command.

\ptitle{Hint:} To calculate the 2's compliment by hand just invert all the bits and add 1.

\subsection{Big-/ Little-Endian}
A given register \code{t1 = [{\color{blue}0xAA},{\color{magenta}0xBB}]} can be represented in memory as

\begin{tikzpicture}

    \node[rectangle,
        text width = 30pt,
        text height = 6pt,
        draw = none,
        text = black,
        fill = gray!20] (r) at (0,0) {};

    \node[rectangle,
        text width = 30pt,
        text height = 6pt,
        draw=none,
        text = black,
        fill = blue!40] (r) at (30pt,0) {\fncode{0xAA}};

    \node[rectangle,
        text width = 30pt,
        text height = 6pt,
        draw=none,
        text = black,] at (30pt,12pt) {\fncode{t1}};

    \node[rectangle,
        text width = 30pt,
        text height = 6pt,
        draw=none,
        text = black,
        fill = magenta!40] (r) at (60pt,0) {\fncode{0xBB}};

    \node[rectangle,
        text width = 30pt,
        text height = 6pt,
        draw = none,
        text = black,
        fill = gray!20] (r) at (90pt,0) {};

    \node[rectangle,
        anchor = west,
        text height = 6pt,
        draw=none,
        text = black,] at (110pt,0pt) {\fncode{DRAM (big-endian)}};

\end{tikzpicture}

\begin{tikzpicture}

    \node[rectangle,
        text width = 30pt,
        text height = 6pt,
        draw = none,
        text = black,
        fill = gray!20] (r) at (0,0) {};

    \node[rectangle,
        text width = 30pt,
        text height = 6pt,
        draw=none,
        text = black,
        fill = magenta!40] (r) at (30pt,0) {\fncode{0xBB}};

    \node[rectangle,
        text width = 30pt,
        text height = 6pt,
        draw=none,
        text = black,] at (30pt,12pt) {\fncode{t1}};

    \node[rectangle,
        text width = 30pt,
        text height = 6pt,
        draw=none,
        text = black,
        fill = blue!40] (r) at (60pt,0) {\fncode{0xAA}};

    \node[rectangle,
        text width = 30pt,
        text height = 6pt,
        draw = none,
        text = black,
        fill = gray!20] (r) at (90pt,0) {};

    \node[rectangle,
        anchor = west,
        text height = 6pt,
        draw=none,
        text = black,] at (110pt,0pt) {\fncode{DRAM (little-endian)}};

\end{tikzpicture}

RISC-V uses little-endian.


\subsection{Conditional branches}
Conditional branches let the PC jump to a specified line relative to the current position if the given condition holds.

\includegraphics[width=0.7\linewidth]{conditional_branches.png}

\begin{itemize}
    \item 12 bits to branch ``around''
    \item Immediate is always multiplied by 2
    \item Branch backward or forward: [PC + 4KB, PC - 4KB]
\end{itemize}

\subsubsection{Basic If-Statement}

\setlength{\oldtabcolsep}{\tabcolsep}\setlength\tabcolsep{2pt}
\begin{tabular}[width=\linewidth]{p{0.48\linewidth} p{0.48\linewidth}}
    C++
    \begin{lstlisting}[escapechar=@]
if(@\aftergroup\magentacodecolor@condition == true@\aftergroup\blackcodecolor@){
    // do something
} else {
    // do something else
}\end{lstlisting}
     &
    RISC-V
    \begin{lstlisting}[escapechar=@]
@\aftergroup\magentacodecolor@bne a0, a1, else_label@\aftergroup\blackcodecolor@
# do something
beq zero, zero, out_label

else_label:
# do something else

out_label:
\end{lstlisting}
\end{tabular}
\setlength{\tabcolsep}{\oldtabcolsep}

\subsubsection{Basic For-Loop}

\setlength{\oldtabcolsep}{\tabcolsep}\setlength\tabcolsep{2pt}
\begin{tabular}[width=\linewidth]{p{0.48\linewidth} p{0.48\linewidth}}
    C++
    \begin{lstlisting}[escapechar=@]
@\aftergroup\olivecodecolor@for@\aftergroup\blackcodecolor@(@\aftergroup\magentacodecolor@i = 0@\aftergroup\blackcodecolor@; @\aftergroup\cyancodecolor@i < size@\aftergroup\blackcodecolor@; @\aftergroup\purplecodecolor@++i@\aftergroup\blackcodecolor@){
    // do something
}
\end{lstlisting}
     &
    RISC-V
    \begin{lstlisting}[escapechar=@]
# a1 = size
    @\aftergroup\magentacodecolor@add t0, zero, zero@\aftergroup\blackcodecolor@
@\aftergroup\olivecodecolor@loop:@\aftergroup\blackcodecolor@
    @\aftergroup\cyancodecolor@bge t0, a1, exit@\aftergroup\blackcodecolor@
    # do something
    @\aftergroup\purplecodecolor@addi t0, t0, 1@\aftergroup\blackcodecolor@
    @\aftergroup\olivecodecolor@beq zero, zero, loop@\aftergroup\blackcodecolor@
exit:
\end{lstlisting}
\end{tabular}
\setlength{\tabcolsep}{\oldtabcolsep}

\subsubsection{Optimized For-Loop}
\setlength{\oldtabcolsep}{\tabcolsep}\setlength\tabcolsep{2pt}
\begin{tabular}[width=\linewidth]{p{0.48\linewidth} p{0.48\linewidth}}
    C++
    \begin{lstlisting}[escapechar=@]
@\aftergroup\olivecodecolor@for@\aftergroup\blackcodecolor@(@\aftergroup\magentacodecolor@arr_end = arr + size@\aftergroup\blackcodecolor@; @\aftergroup\cyancodecolor@arr++ < arr_end@\aftergroup\blackcodecolor@;){
    // do something
}
\end{lstlisting}
     &
    RISC-V
    \begin{lstlisting}[escapechar=@]
# a0 = arr
# a1 = size
    @\aftergroup\magentacodecolor@add a1, a0, a1@\aftergroup\blackcodecolor@
@\aftergroup\olivecodecolor@loop:@\aftergroup\blackcodecolor@
    @\aftergroup\cyancodecolor@bge a0, a1, exit@\aftergroup\blackcodecolor@
    # do something
    @\aftergroup\purplecodecolor@addi a0, a0, 1@\aftergroup\blackcodecolor@
    @\aftergroup\olivecodecolor@beq zero, zero, loop@\aftergroup\blackcodecolor@
exit:
\end{lstlisting}
\end{tabular}
\setlength{\tabcolsep}{\oldtabcolsep}

\subsection{Unconditional Jumps}

\begin{itemize}
    \item \texttt{JAL} is a UJ-type and can jump up to $2^{20}$ 2-byte instructions relative to the PC
    \item \texttt{JALR} is an I-type and can jump anywhere in memory (independent of the PC)
\end{itemize}

\includegraphics[width=0.7\linewidth]{unconditional_jumps.png}
\begin{itemize}
    \item \texttt{JAL} stores the address of the instruction following the jump (PC+4) into the register \texttt{rd}.\newline
        The immediate is multiplied by 2 (64-bit compatability)
    \item The standard software calling convention uses \texttt{x1/ra} as the return address register.
    \item The registers that are used after the PC returned to the line below the jump instruction have the be saved to the stack.
    \item The table~\ref{registers} shows all the registers that have to be saved by the \textbf{caller/callee}.
\end{itemize}


\subsubsection{Basic function call}

\begin{lstlisting}[language={[RISC-V]Assembler}]
# a0 = entry point of a function 
addi    sp, sp, -4      # allocate memory on stack
sw      ra, 0(sp)       # store return address on stack

jalr    ra, a0          # jump to the function at a0
# or
jal     ra, funtion_label

lw      ra, 0(sp)       # load return address from stack 
addi    sp, sp, 4       # reset the stack pointer
\end{lstlisting}

\subsection{Useful Instructions}
RISC-V instruction set architecture (isa) \code{RV32I} can be found in Section\ \ref{riscv_isa}.

\newpar{}
\ptitle{Bitwise Invertion}

\begin{lstlisting}[language={[RISC-V]Assembler}]
addi    a1,zero, -1         # create 0b11...111
xor     a0,a0,a1            # xor with 0b11..11
\end{lstlisting}

\newpar{}
\ptitle{64-bit extension}

\begin{lstlisting}[language={[RISC-V]Assembler}]
slli    a0,a0,0x3f      
srai    a0,a0,0x3f          # 64 bit extended mask of a0
\end{lstlisting}

\newpar{}
\ptitle{Load and Store with Offset}

\begin{lstlisting}[language={[RISC-V]Assembler}]
# a0 = base address
# a1 = offset
slli    a1,a1,0x2           # multiply offset by 4 (byte) 
                            # <- 32-bit integer
add     a2,a0,a1            # offset address

lw      a3,0(a2)            # load from offset address
addi    a3,a3,0x1           # increment by one
sw      a3,0(a2)            # store to offset address
\end{lstlisting}

\newpar{}
\ptitle{Find element in array}

Find the first \texttt{0} in the 32-bit (4-byte) integer table starting at \texttt{a0} and return the number of elements.
\begin{lstlisting}[language={[RISC-V]Assembler}]
    add     t0, zero, zero      # initialize count
Loop:

    lw      t1, 0(a0)           # load integer to t1
    beq     t1, zero, exit      # check if t1==0
    addi    a0, a0, 4           # increment address
    addi    t0, t0, 1           # increment count

    beq     zero, zero, loop
Exit:
    add     a0, zero, t0        # return count
\end{lstlisting}


\newpar{}
\ptitle{Recursive Function}

When calling a function or when using recursion, both the return address \code{ra} and the saved register \code{s0-s11} have to be stored on the stack, and restored when exiting.
\begin{lstlisting}[language={[RISC-V]Assembler}]
rec_fun:
    # Pre
    addi    sp, sp, -16     # increase stack pointer
    sw      s0, 0(sp)
    sw      ra, 4(sp)       # save the stack pointer into the return address

    # check if a1 >= 1
    add     t0,zero,1
    bge     a1,t0,Else
    add     a0,zero,zero
    beq     zero,zero,Exit  # end of if

Else:
    # call f(a0 = a1-1)
    add     a0,a1,-1        # a0 = a1-1
    add     s0,zero,a1      # save a1 for later
    jal    ra, rec_fun     # jump to the function, ra = PC+4

    # f(a1-1) is now stored in a0
    add     a0,s0,a0

Exit:
    # Post
    lw      s0, 0(sp)       
    lw      ra, 4(sp)
    addi    sp, sp, 16       # reset/decrease the stack pointer

\end{lstlisting}

% \newpar{}
% \ptitle{Matrix multiplication}
% 
% $M0, M1, M2$ are 16-bit (2-byte) integer matrices.
% 
% $M0\cdot M1=M2$ with $M0\in\mathbb{Z}^{a2\times a1}$, $M1\in\mathbb{Z}^{a1\times a4}$ and $M2\in\mathbb{Z}^{a2\times a4}$
% 
% \begin{lstlisting}[language={[RISC-V]Assembler}]
% add t3, zero, a0        # M0 ptr
%     add t4, zero, a3        # M1 ptr
%     add t5, zero, a5        # M2 ptr
%     addi t6, zero, 2        # offset
% 
%     # i = t0
%     add t0, zero, zero      
%     loop_i:
%         bge t0, a2, exit_i
% 
%         # j = t1
%         add t1, zero, zero      
%         loop_j:
%             bge t1, a4, exit_j
% 
%             # set M0 start index
%             mul t3, a1, t6
%             mul t3, t3, t0
%             add t3, t3, a0
%             
%             # set M1 start index
%             mul t4, t1, t6 
%             add t4, t4, a3
% 
%             # set current M2 cell to 0
%             add s5, zero, zero
%             
%             # k = t2
%             add t2, zero, zero
%             loop_k:
%                 bge t2, a1, exit_k
%             
%                 # load the M0 and M1 value and multiply them
%                 lh s2, 0(t3)
%                 lh s3, 0(t4)
%                 mul s2, s2, s3
%                 # add the product to the row sum
%                 add s5, s5, s2
%             
%                 # set next M0 index
%                 add t3, t3, t6
%             
%                 # set next M1 index
%                 mul s6, a4, t6
%                 add t4, t4, s6
%             
%                 # increment k
%                 addi t2, t2, 1
%                 beq zero, zero, loop_k
%             exit_k:
%             
%             # save result to M2
%             sh s5, 0(t5)
% 
%             # move M2 pointer to next cell
%             add t5, t5, t6              
% 
%             # increment j
%             addi t1, t1, 1
%             
%             beq zero, zero, loop_j
%         exit_j:
% 
%         # increment i
%         addi t0, t0, 1
% 
%         beq zero, zero, loop_i
%     exit_i:
% \end{lstlisting}
