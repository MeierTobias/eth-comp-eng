\section{Multiprocessing}
\subsection{Process Data structure}
A central data structure (``task'' in Linux) that keeps track of the processes' resources:
\begin{itemize}
    \item Page tables
    \item VMAs
    \item Registers (e.g.\ during exceptions)
\end{itemize}
The data structure gets updated when the process
\begin{itemize}
    \item Is first created
    \item Allocates/frees resources (e.g.\ heap)
    \item Gets scheduled in/out
    \item Dies
\end{itemize}

\newpar{}
\ptitle{Process Data structure in Jake}
\begin{lstlisting}[style=bright_C++]
struct process {
    unsigned int pid;
    enum PROCESS_STATUS status; 
    struct node *uvmas;         // user VMAs
    struct trap_frame tf;       // user registers
    unsigned long satp;         // process' root PT
    unsigned long heap_end_vpn;
    struct elf_jake elf;        // where program instructions are stored
    };
\end{lstlisting}
\newpar{}
\begin{lstlisting}[style=bright_C++]
enum PROCESS_STATUS {
    PROCESS_FREE_SLOT, 	    // Slot free to be used
    PROCESS_RUNNING, 	    // Usermode runs process
    PROCESS_PAUSED, 	    // Process will be scheduled to run in usermode
    PROCESS_RESERVED,
};
\end{lstlisting}

\subsection{Starting a New Process}
\subsubsection{Executable Files}
\begin{itemize}
    \item Executables created by compiler
    \item ELF: Linux, Jake
    \item Portable Executable (PE): Windows
    \item Common OS: executables stored in SSD, link to OS
    \item Jake: stored directly in kernel (no file system): need to know all applications at kernel compilation
\end{itemize}
\newpar{}
\ptitle{Application Linking in Jake}
%%% relevant? %%%
\newpar{}
\ptitle{Executable Linkable Format (ELF)}
\begin{center}
    \includegraphics[width = .5\linewidth]{ELF.png}
\end{center}
Defines where \code{.text}/\code{.data}/\code{.bss} sections
\begin{itemize}
    \item are located inside the executable file
    \item should be loaded in (virtual) memory
    \item \code{.text} entry address would define the first instruction of the program
\end{itemize}

\newpar{}
\ptitle{Preparing \code{.text} VMA in Jake}

After starting the new process, its first instruction will cause a page fault (no mapped yet). Since Jake does not have access to the user-mode virtual memory, it uses \textit{physmap} instead which allows for modification to page tables for creating new virtual mappings by mapping the entire physical memory to virtual memory.
% TODO: I think here we want to describe the copy process via the identity mapping. I would move this to the multi process section and not in the executable file section. Also the next chapter is "Starting the First Process" which is also out of order if this stays here.

\subsubsection{Starting the First Process}
Common OS start special first process (init process) which starts other processes. First, the specific process data structure must be initialized and then, one switches to user mode.
\ptitle{Process Data Structure}

On start-up, the process data structure gets populated as follows:
\begin{itemize}
    \item Select a pid
    \item Allocate root PT. Store this physical address in \texttt{satp} field
    \item Copy the kernel page tables of the process
    \begin{itemize}
        \item Kernel has identical virtual address for each process but translation is process-specific (\texttt{satp})
    \end{itemize}
    \item Initialize the VMAs (code and stack)
    \item Parse the \texttt{elf} datastructure
\end{itemize}

\newpar{}
\ptitle{Switching to User Mode}
\begin{center}
    \includegraphics[width = \linewidth]{switch_user_mode.png}
\end{center}
To start a process, the kernel needs to know
\begin{itemize}
    \item The \code{.text} section of the application (where it lies)
    \item The address of 1st instruction
\end{itemize}
One can then switch to usermode after
\begin{itemize}
    \item Initializing user mode registers to zero (hide kernel data used before \texttt{sret})
    \item Set SEPC to first instruction (we get it from \texttt{elf.elf.e\_entry})
    \begin{lstlisting}[style=bright_C++]
    asm volatile("mv t0, %0\n\t"
            "csrw sepc, t0\n\t" ::"r"(elf.elf.e_entry)
            : "t0");
    \end{lstlisting}
    \item Set user-mode SP (can happen in kernel (e.g.\ Jake) or user-mode)
    \begin{lstlisting}[style=bright_C++]
    asm volatile("mv a0, %[farg]\n\t"
            "mv sp, %[stack]\n\t"
            "li ra, 0\n\t"
            "li t0, 0\n\t"
            .
            .
    \end{lstlisting}
    \item call \code{sret} with \code{sstatus.spp = 0}
\end{itemize}
% TODO: Are the code examples really needed here. I think they just show a specific case and not a general idea.

\subsubsection{Multiprocessing}
Applications should be able to create new processes. Most kernels provide a system call called \texttt{fork}:

\newpar{}
\ptitle{Fork System Call}

Fork creates an \textbf{identical copy} of the address space of the calling process. The process calling fork is called \textit{parent}. Fork returns either 0 or 1 to distinguish whether one has ended up in the child or parent process.
\begin{center}
    \includegraphics[width = 0.8\linewidth]{fork.png}
\end{center}

Just calling \texttt{fork} can be useful sometimes (e.g.\ running same process on multiple CPUs) but to create a \textit{new} process one must call \texttt{exec}:

\ptitle{Exec System Call}
\begin{itemize}
    \item Typically called in the \textbf{child right after a fork}
    \item Replaces VMAs of child.
    \item For \code{.text} VMA, it takes an executable file (e.g., ELF)
\end{itemize}

\newpar{}
\ptitle{Example: Combined System Call in Jake}

In jake, \code{fork} and \code{exec} are combined to \code{execv}
% TODO: Can we remove this example? This is again a very specific example for jake and not really connected to the chapter (command execv)
\begin{lstlisting}[style=bright_C++]
// a1: identifies next application to be run
// a2: main() argument for next process
// a3: return value
void scall_handler_execv()
{
    process_list[proc_running].status = PROCESS_PAUSED;     // pause current process
    tf_user->a0 = 0;    // return success to parent                                    
    proc_copy_frame(&process_list[proc_running].tf, tf_user);

    void *ptr__ = NULL;
    // next application to be run
    switch (tf_user->a1) {
    case 0:
        ptr__ = &_testing;
        break;
    case 1:
        ptr__ = &_hello_jr;
        break;
    default:
        printerr("ERROR: Non-existing application.\n");
        ecall_poweroff();
        break;
    }
    // call next app (won't return on success!)
    if (proc_run_process(ptr__, tf_user->a2, tf_user->a3) < 0) {
        // failed creating new process
        tf_user->a0 = -1;
        csrw(satp, process_list[proc_running].satp);
        pt_flush_tlb();
        process_list[proc_running].status = PROCESS_RUNNING;
    }
}
\end{lstlisting}

\subsubsection{Jake Implementation}
% TODO: Should we add the code examples?
% Please NO
% Also a NO from me ;) (as commented I would remove all the code that can not be expressed as pseudo-code.)

\subsection{CPU Virtualization}
After creating the compatibility for running multiple processes one can virtualize the CPU. The idea of CPU virtualization is similar to memory virtualization:
\begin{itemize}
    \item Each process should have the illusion of having the entire CPU (kernel should create the illusion of infinite CPUs despite having only one or few)
    \item Software mechanism: context switch
    \item Policy: process scheduling
\end{itemize}
\subsubsection{Context Switching}
The designation ``context'' of a process means
\begin{itemize}
    \item Value of the registers
    \item Address space (described by VMAs)
    \item Page tables and physical pages (reachable through satp)
\end{itemize}
As VMAs and physical memory are preserved, only the CPU registers must be saved at context switch.

\begin{itemize}
    \item The process' context is \textbf{stored} in the \code{trap frame} data structure inside the process data structure.
    \item CPU cache needs to be flushed (contains old address space)
\end{itemize}

\subsubsection{Scheduling}
\paragraph{Round-Robin (RR) Scheduling}
\ptitle{Cooperative RR}

Go through the processes one by one and give each a chance to execute. In jake, one just walks the process list to find the next feasible process to implement RR.\\
Starvation in cooperative RR:
\begin{center}
    \includegraphics[width = \linewidth]{RR_motivation.png}
\end{center}
Some processes that only require short CPU time might suffer starvation (blue process).

\newpar{}
\ptitle{Priority-Based RR}

Idea of automated prioritization: kernel learns the processes' priorities. Every process starts at a high-priority priority, the kernel moves it to lower priorities if it uses the CPU too much.\\
Starvation in priority-based RR:
\begin{center}
    \includegraphics[width = \linewidth]{RR_prio_starvation.png}
\end{center}
To avoid this, one moves all jobs to high-priority queue once in a while.

\paragraph{Scheduling Modes}
% The kernel needs to run to be able to make a scheduling decision.

\newpar{}
\ptitle{Cooperative (Non-Preemptive) Scheduling}

\begin{itemize}
    \item High-performance computing usually uses cooperative scheduling (less overhead).
    \item System call for cooperative scheduling: \code{yield}.
    \item On \code{yield}, the kernel performance a context switch to another process: % TODO: The details are already mentioned a few times. Can we remove this
          \begin{enumerate}
              \item Store the registers in the old process trap frame
              \item Load the registers of the new process from its trap frame
              \item call \code{sret}
          \end{enumerate}
\end{itemize}
Pros/cons:
\begin{itemize}
    \item[+] Improved process performance
    \item[+] Simple
    \item[-] Process can \textbf{hog} CPU
    \item[-] Process is unaware of other processes: locally and globally suboptimal
\end{itemize}

\newpar{}
\ptitle{Non-Cooperative (Preemptive) Scheduling}
\begin{itemize}
    \item Standard systems (e.g., phones, laptops, etc.) have non-cooperative scheduling (better responsiveness)
    \item Uses a timer connected to CPU to periodically generate \textbf{interrupts} (same as exception or trap)
    \item Kernel periodically preempts processes on each interrupt
    \item Exception handling: kernel checks on interrupt whether a context switch is necessary and executes it if
\end{itemize}
Pros/cons:
\begin{itemize}
    \item[+] Process can't \textbf{hog} CPU
    \item[+] Increased responsiveness
    \item[-] Might cause negative performance impact
    \item[-] More complex
\end{itemize}

\paragraph{Timers}
External oscillator attached to CPU

\newpar{}
\ptitle{Memory-Mapped I/O (MMIO)}

MMIO enables interaction with devices through memory accesses: access \textbf{device rather than DRAM}.

\newpar{}
\ptitle{RISC-V QEMU Timers}

\begin{itemize}
    \item Current QEMU version only supports machine-mode timers
          \begin{itemize}
              \item Timers can only be configured in machine mode
              \item The interrupts will \textbf{always} be delivered to machine mode (non-maskable)
          \end{itemize}
    \item We need to mret to supervisor mode for scheduling
\end{itemize}
\begin{center}
    \includegraphics[width = \linewidth]{QEMU_timer.png}
\end{center}

